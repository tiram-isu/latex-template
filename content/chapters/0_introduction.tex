\chapter{Introduction}

\section{Structures}

\subsection{asdasd}
asdfasdf

\subsubsection{Subsubsection}
asdf

\paragraph{Paragraph}
asdf

\section{Example Figures}

\lipsum[10]

\begin{wrapfigure}{rt}{8cm}
\caption{Figure Caption}
\centering
\includegraphics[width=0.3\textwidth]{content/pictures/hfu}
Quelle: \cite{s11wasml}
\label{pic:bild2}
\end{wrapfigure}

\lipsum[10]

\begin{figure}
\caption{Figure Caption}
\includegraphics[width=1\textwidth]{content/pictures/hfu}
Source: \cite{s11wasml}
\label{pic:bild1}
\end{figure}

\section{Example Table}

\begin{table}
\caption{Table Caption}
\center
\footnotesize
\begin{tabular}{lll}
\toprule
Head1 & Head2 & Head3 \\
\midrule
Val1 & Val2 & Val3 \\
Val4 & Val5 & Val6 \\
\bottomrule
\end{tabular}
\end{table}

\section{Listings}

\begin{lstlisting}[language=java, caption=Hello World in Java]
public class HelloWorld 
{
	public static void main(String[] args) 
	{
		System.out.println("Hello World!");
	}
}
\end{lstlisting}

%\mylisting{from}{to}{language}{file}{descr}{path}
\mylisting{1}{9}{java}{Java}{Everything}{content/code/example.java}
\mylisting{6}{7}{java}{Java}{Excerpt}{content/code/example.java}

\section{Abbreviations}

Abbreviations listed in the abbreviation list can be specified in long form (\ac{HFU}) or short form (\acs{HFU}).

\section{Set-off Direct Quotation}

\lipsum[10]

\begin{quote}
\textit{\enquote{Indented direct quotation.}}\cite[S. 14ff]{s11wasml}
\end{quote}

\lipsum[10]

\section{Theorems}

\lipsum[2]

\begin{example}
Example text\dots
\end{example}
 
\lipsum[2]

\begin{thesis}
Thesis\dots
\end{thesis}
 
\lipsum[2]
 
\begin{definition}
	By the term \dots we understand \dots
\end{definition}

\lipsum[2]